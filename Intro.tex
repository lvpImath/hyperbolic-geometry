\chapter*{\begin{center}
LỜI NÓI ĐẦU
\end{center}} 
\addcontentsline{toc}{chapter}{{\bf{Lời nói đầu}}\rm }
Vào giữa thế kỷ 18, \textit{Gauss} tuyên bố đã khám phá ra hình học mới trong cuốn sách chưa xuất bản của mình mang tên \textit{Hình học phi Euclide} trên mô hình nửa mặt phẳng trên. Vào khoảng năm 1830, \textit{Nikolai Lobachevsky} đã xuất bản hệ thống hoàn chỉnh của hình học hyperbolic , nơi ông đã thay đổi tiên đề song song bằng cách phát biểu rằng, qua một điểm nằm ngoài một đường thẳng có vô số đường thẳng song song với đường thẳng đó.
Vào thế kỷ 19, các nhà toán học bắt đầu nghiên cứu hình học hyperbol một cách rộng rãi và nó vẫn đang được nghiên cứu trên thế giới một cách tích cực. 

Trong khoá luận này chúng ta sẽ mở đầu tìm hiểu về hình học hyperbolic trên mô hình nửa mặt phẳng hyperbolic $\hh$. Trên đó, ta chú tâm đến nhóm các đẳng cự bảo toàn hướng $\PSL(2,\R)$ và đặc biệt là các nhóm con rời rạc của nó.

Bố cục của khóa luận bao gồm 3 chương:
\begin{itemize}
\item Chương 1 của khóa luận trình bày tóm tắt về các kiến thức chuẩn bị và các định lý, bổ đề sẽ được sử dụng trong chứng minh của bài toán. 
\item Chương 2 của khóa luận đi vào trình bày về hình học hyperbolic trên mô hình mặt phẳng hyperbolic.
\item Chương 3 của khóa luận đi vào trình bày về nhóm con rời rạc của nhóm các phép đẳng cự bảo toàn hướng của mặt phẳng hyperbolic.
\end{itemize}
Do thời gian thực hiện khóa luận không nhiều, kiến thức còn hạn chế nên khi làm khóa luận không tránh khỏi những sai sót.
Em mong nhận được sự góp ý và những ý kiến phản biện của quý thầy cô và bạn đọc. 

Em xin chân thành cảm ơn!